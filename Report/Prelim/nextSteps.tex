\chapter{Next Steps}\label{C:nextSteps}

\par A custom Intellectual Property block for the timing functions needs to be created. This will measure the time taken between two ethernet ports, and store the value into a register which is accessible by the ARM core present on the FPGA. This allows for ease of accessing through a Linux subsystem.
I am aiming for this to be done within 2 weeks into trimester 2. This is because I should theoretically have a low workload from my papers in the first few weeks, and I can focus on the project more than anything else.
Once the initial design is ready for testing, I can deploy it to a device and test with real hardware to see how well it can measure timings between packets.
I feel the need to speed up this work as I will have a lot of work to do from my other papers as they are very practical based papers.

\par Once the timing system is functional, tests will need to be done on network switches to measure the packet timing.
Analysis will then be done on the information to understand how some network switches reduce latency compared with others. 
This analysis of the data will be ensuring the reliability and repeatability of the tests. 
After tests and analysis has been performed, work shall begin on the final report and presentation.
I am aiming for this to happen in the second half of the trimester.
