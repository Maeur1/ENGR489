\chapter{Design}\label{C:design}

\section{Electronic Signalling}

\par Latency at an electronic signal level is the accurate way of measuring the time cost to jump between nodes. 
When electronic signals are received by a computer, there is a delay between the processing of the software 
in the computer and the registering of the event in the CPU of the computer. This delay can vary based on load 
on the computer, how the computer is operating and other external factors. To remove these factors, the decision 
was made to measure latency of when the electronic signals leave the initial computer, and are detected at the 
destination.

\section{Ethernet}

\par Ethernet was chosen as the platform to base the project around, as it is used widely across many network 
infrastructures locally. The platform is very mature, as it has been used for many years, and a higher performance 
version is used in data centres and core networking. This higher performance version is very similar to the Gigabit 
version of Ethernet, but involves a higher frequency when clocking out data, but the signalling is very similar. 
This means work done in this project can be used as a stepping stone to applying similar practices on higher 
performance networking applications.

\section{Discrete Logic}

\par The clocking signals provided from Ethernet PHYs are electronic signals which combine logic operations to transmit 
and receive data. This could be measured using discrete logic, as the means to interface with high speed busses 
ranging in the MHz clocking region exists (Potato Semiconductors). The major issue is that measurement information 
is needed to be provided in the digital workable form, for further processing.  This means producing a large binary 
counter which can be accessed by a computer. Doing so would be very complex and a tedious task to design such a 
large system, both physically and time wise.

\section{Microcontroller}

\par A process of measuring signals accurately, is to setup interrupts on a microcontroller, to precisely detect events 
on a wire.  This involves halting the CPU to do an interrupt service routine, and do a set task (such as read a 
timing register) and then continue with the original interrupted program. The time the CPU takes to register the 
interrupt and save a timer value is limited to the maximum frequency that the CPU can run at. This has a maximum of 
204 MHz, making the minimum time that can be measured greater than 4.9 ns. Interfacing with digital systems to 
provide easily accessible measurements through SD cards, or other commonly used data transfer medium.

\section{FPGA}

\par A FPGA is suited to high speed, high performance tasks. It is a way to combine complex digital systems together in 
a small physical footprint. As the FPGA uses logic elements to do all the signal processing, there is very little 
latency introduced through components in the design. This latency reduction ensures that the measured timing is as 
accurate as possible. The FPGA includes the flexibility to communicate and interact with complex digital protocols 
while also able to perform time critical tasks. With this added flexibility comes compromise, where the programming 
requires a proprietary procedure unique to each FPGA manufacturer, and limited capabilities with the resources 
provided in each FPGA IC. This is combated with the use of a microprocessor interconnected to the FPGA, meaning the 
complex workloads of simple digital tasks are offloaded to the microprocessor for post processing.
