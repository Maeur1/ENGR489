\chapter{Introduction}\label{C:intro}

\par Fast internet connections are becoming more and more desired as high bandwidth media consumption and internet 
based services grow in popularity. Speed of an internet connection can be separated into two distinct metrics, 
latency and bandwidth. Latency is the time taken for information to travel from one place to another and the delay 
introduced by the device itself, while bandwidth is the amount of information that can flow reliably from end to end.
When a user is accessing a website, such as Facebook, their requests traverse multiple network devices (physical 
hardware) to reach that website's servers. Throughout this travel, each network device introduces some latency, as 
there is processing time required to send information to the right destination. Although this delay is very small, 
over the course of multiple hops it can become significant. Measuring this small latency with current solutions is
convoluted and without the right systems, can be unreliable. By measuring this latency, we can enable research to
reduce latency to create a faster internet.

\section{Problem}

\par There has been research dedicated to reducing latency. Some ways to reduce this latency include making “smart” network routers \cite{smartrouters} and low latency wireless systems \cite{5g}. 
This research is reaching the point where network traffic can have latency as little as 1ms \cite{lessthan1ms} or below.
To advance further, new tools need to have the capability of measuring this latency.

\par Existing hardware solutions to measuring latency are capable of measuring less than 1 ms but require some 
additional training in the software suite provided with the hardware \cite{dagfeatures}. Additional processes attached
to hardware based methods can require dedicated computers for the measurement and analysis of networks. This means 
that each researcher would require an understanding of the software suite accompanied with a dedicated computer 
all setup for specific tests to be run. Each different test would require reconfiguring of both hardware and software.
This drives the cost of using the device up, in terms of both time and money.

\par Current software solutions can measure latency in ethernet network traffic down to a resolution of
milliseconds and sometimes microseconds \cite{pingisbad}. There are no software solutions that can measure
latency in the order of nanoseconds reliably \cite{timeinlinux}. Measuring the latency in nanoseconds is needed as
network switches have very low latency and the time between ingress and egress can range from
microseconds to nanoseconds. If it could be measured accurately, then analysis could be performed
on network traffic through network switches to research new ways to reduce latency.

\par Existing methods to measure latency within microseconds and nanoseconds require monolithic devices which need 
setting up and installing device drivers, proprietary software, or even a replacement of existing hardware systems.
Once setup, these devices often require data conversion of results to be compatible with data analysis methods and 
evaluation programs. Occasionally these programs are embedded within in manufacturers software, but the user is 
limited to what the manufacturer has chosen as relevant outputs for the user. These can be useful most of the time, 
but occassionally the user would need to export the data to another data analysis tool they are familiar with for 
futher processing. This inflexibility is prominent in many high end network performance analysis tools, causing 
delays in retrieving test results for evaluation.

\par A reduction in network latency would be a huge boon to future technologies. For example, less
latency in video streaming can enable doctors from remote locations to perform surgery with
minimal lag between actions and responses \cite{remotesurgery}. The stock market is another example of improvement with
this technology, as less latency ensures that stock market trade deals can be completed faster.

\begin{quote}
    \centering
    ``We are running through the United States with dynamite and rock saws, so that an algorithm can
    run three microseconds faster.`` \em --- Kevin Slavin, How Algorithms shape our world \cite{tedTalkAlgorithms}
\end{quote} 

\par Many improvements made at a nanosecond scale can add up to microseconds or even milliseconds of latency
reduction. Improvements cannot be done without knowing how much this can improve by without the tools to measure it.

\par A software based approach has the limitation that the timing functions must be processed by a
Central Processing Unit (CPU) \cite{CPUtiming}. This means the counters for timing functions could be offset from the true value. Higher
resolution latency measurements can be achieved through a hardware implementation of a packet
timer. This circumvents the limitations of a software based approach.

\section{Solution}

Latency occurs the instant that a network packet is sent from a computer. The processing times for the network packet
to flow through the networking stack in the operating system can be considered a constant unavoidable delay which is
not in the scope of this project.

Because the nature of when the latency occurs, it would be best measured by considering the raw electronic signals 
from the ethernet port on the device to ensure there is no operating system lag. To achieve this, microcontrollers, 
discrete logic gates and a Field Programmable Gate Array could be implemented. A microcontroller could be used but 
then the same issues arise, that a CPU clocking through instructions would deviate the true timing counter value. 
Another approach would be to use discrete logic gates to measure the timing accurately, but this will become 
complicated very rapidly, and propagation delays through discrete devices would reduce the accuracy. Hence the 
preferred choice is to implement the packet latency measurement unit on a Field Programmable Gate Array (FPGA). More 
information about FPGA’s is in the Background section of the report. This way complex digital designs can be 
simplified down to a simple set of blocks, and there is no overhead that a processor would introduce. The drawback 
to using an FPGA is that the learning curve to developing on the platform is quite steep, and also communication 
with existing systems is difficult and is usually aided by the use of a microcontroller (such as an ARM or x86 based 
co-processor)

\par This report will focus on the work done so far in implementing a FPGA based packet latency timing
device. This project is limited to Ethernet based networks as they are most widely used \cite{etherneteverywhere}.
