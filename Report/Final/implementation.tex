\chapter{Implementation}\label{C:impl}

Begin with generating a packet, and how it generates the RGMII signals

Then Flow into how the MAC addresses are set in the data section of the packet

Then talk about generating an interrupt when sent

Then talk about how to detect an incoming packet on the other port and generating an interrup from that

Then talk about Measuring the time between the two interrupts. 

Then talk about how to get the value from the register which stores the interrupt timer.

Then talk about the math to convert to ns.

Then talk about how to write it to the SD card. bring in some SDK screen shots of how to setup xilffs

\lstinputlisting[language=VHDL, firstline=36, lastline=37, basicstyle=\small]{Code/byte\string_data.vhd}

\lstinputlisting[language=VHDL, firstline=46, lastline=52, basicstyle=\small]{Code/edge\string_detector.vhd}

\lstinputlisting[language=VHDL, firstline=57, lastline=70, basicstyle=\small]{Code/InterruptTimer.vhd}

\lstinputlisting[language=VHDL, firstline=86, lastline=92, basicstyle=\small]{Code/packet\string_timer.c}

\lstinputlisting[language=VHDL, firstline=311, lastline=343, basicstyle=\small]{Code/gigabit\string_test.vhd}
