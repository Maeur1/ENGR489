\chapter{Implementation}\label{C:impl}

\section{Suitable Development FPGA Board}

\par Selecting an FPGA requires the use of a development board. Although custom boards can be developed for a 
specific purpose, a development board can have an assortment of peripherals and required components needed to 
have a functioning device to program. 

\subsection{Required Accompanying Hardware}

\par For the chosen FPGA family, the embedded processor requires some external components to function correctly.

\subsubsection{Flash Memory}

\par Flash memory is a solid state electronic memory storage medium. This is needed by the microprocessor in 
the FPGA to configure the FPGA from a configuration file saved on the memory. This can also store an operating 
system for the microprocessor to work with, or a boot program to execute when the board is powered up. A common 
interface between the FPGA microprocessor and Flash memory is through QSPI, which is an implementation of SPI with 
4 times the bandwidth through extra lanes provided. 

\subsubsection{Random Access Memory (RAM)}

\par The microprocessor built into the FPGA needs fast memory location to access large amounts of data. RAM is 
connected to the FPGA to provide an ultra-fast (1.3 GHz) access to data, ensuring that the CPU is not held in an 
idle state when processing data. This is a common design with modern day processors and computers.

\subsubsection{Universal Asynchronous Receiver/Transmitter (UART)}

\par Input and Output of the microprocessor is connected to two physical wires of the FPGA, which produce and 
receive signals which conform to the UART standard. This allows for ASCII characters to the sent and received by 
the processor. This is useful for debugging a program that is running on the processor, or giving a user some 
feedback on what stage of the program has ran. This protocol is common in electronics, and it is common to obtain 
UART to USB transceivers to be able to send and receive data to USB compatible computer.

\subsubsection{Ethernet Physical Transceiver (Ethernet Phy)}

\par A core function of the project is to be able to send and receive data on a ethernet based network. An Ethernet 
Phy is an integrated circuit (IC) which receives data using a predefined electronic protocol interface and converts 
the data into another protocol. This also works in the opposite direction, as the IC commonly comes in a transceiver 
package. The one used in this project converts RGMII into the differential pair signal found in ethernet. 

\section{PicoZed FPGA Development Board}

\par Avnet is a manufacturer of electronic compoenents and provides electronic design services. A main focus of the company
is networking electronics. A product of Avnet is the family of PicoZed FPGA development kits.

\subsection{PicoZed System on a Module (SOM)}
\subsection{PicoZed FMC Carrier V1}
\section{Zedboard FPGA Development Board}
\subsection{Ethernet FMC Card}
\section{Xilinx Software Suite}
\subsection{Vivado 2017.2}
\subsection{Xilinx Software Development Kit}
\section{Hardware Descriptor Language (HDL)}
\subsection{VHDL}
\subsection{Verilog}
\section{Xilinx University Program (XUP)}
\subsection{FPGA Design Flow}
\subsection{Linux}
\subsection{Embedded}
\subsection{Advanced Embedded}
\section{Block Design}
\subsection{Zynq Processing System (PS)}
\subsection{Zynq Programmable Logic (PS)}
\subsection{Advanced Microcontroller Bus Architecture (AMBA)}
\subsection{Advanced eXtensible Interface (AXI)}
\subsection{Clock Generation}
\subsubsection{Phased Locked Loop}
\subsection{Packet Generator}
\subsubsection{RGMII Signal Processing}
\subsection{Packet Detector}
\section{Software Development Kit}
\subsection{Reading AXI Registers}
\subsection{SD Card}
\subsection{Retrieving Test Results}
