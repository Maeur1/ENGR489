\chapter{Implementation}\label{C:impl}

\section{Suitable Development FPGA Board}

\par Selecting an FPGA requires the use of a development board. Although custom boards can be developed for a 
specific purpose (REFERENCE), a development board can have an assortment of peripherals and required components needed to 
have a functioning device to program (REFERENCE). 

\subsection{Required Accompanying Hardware}

\par For the chosen FPGA family, the embedded processor requires external components to function correctly.

\subsubsection{Flash Memory}

\par Flash memory is a solid state electronic memory storage medium. This is needed by the microprocessor in 
the FPGA to configure the FPGA from a configuration file saved on the memory. This can also store an operating 
system for the microprocessor to work with, or a boot program to execute when the board is powered up. A common 
interface between the FPGA microprocessor and Flash memory is through QSPI(REFERNECE), which is an implementation of SPI with 
4 times the bandwidth through extra lanes provided. 

\subsubsection{Random Access Memory (RAM)}

\par The microprocessor built into the FPGA needs fast memory location to access large amounts of data. RAM is 
connected to the FPGA to provide an ultra-fast (1.3 GHz) access to data, ensuring that the CPU is not held in an 
idle state when processing data. This is a common design with modern day processors and computers.

\subsubsection{Universal Asynchronous Receiver/Transmitter (UART)}

\par Input and Output of the microprocessor is connected to two physical ports (REFERENCE) of the FPGA, which produce and 
receive signals which conform to the UART standard. This allows for ASCII characters to the sent and received by 
the processor. This is useful for debugging a program that is running on the processor, or giving a user some 
feedback on what stage of the program has ran. This protocol is common in electronics, and it is common to obtain 
UART to USB transceivers to be able to send and receive data to USB compatible computer.

\subsubsection{Ethernet Physical Transceiver (Ethernet Phy)}

\par A core function of the project is to be able to send and receive data on a ethernet based network. An Ethernet 
Phy is an integrated circuit (IC) which receives data using a predefined electronic protocol interface and converts 
the data into another protocol (REFERENCE). This also works in the opposite direction, as the IC commonly comes in a transceiver 
package. The one used in this project is the (EHTERHENT PHY HERE) which converts RGMII into the differential pair signal 
found in ethernet. 

\section{Zedboard FPGA Development Board}
\subsection{Ethernet FMC Card}
\section{Xilinx Software Suite}
\subsection{Vivado 2017.2}
\subsection{Xilinx Software Development Kit}
\section{Hardware Descriptor Language (HDL)}
\subsection{VHDL}
\subsection{Verilog}
\section{Xilinx University Program (XUP)}

\par Xilinx University Program provides FPGA workflow tutorials which teach the process of constructing a basic FPGA 
design in the Xilinx Vivado Software suite. The tutorials are targeted at development on a Zedboard FPGA development 
board.

\subsection{FPGA Design Flow}

\par The FPGA design flow tutorials construct a basic understanding of the flow to run a design on a Zedboard. 
This tutorial was to create a simple logic circuit to connect the on board switches to the on board LEDs. 
The tutorial showed how the FPGA can be used to create logic circuits and how input and output is controlled.
This is very different to commonly found microcontroller programming, as the FPGA is only a set of reconfigurable
logic blocks, with no instructions to clock through.

\subsection{Linux}

\par A special feature of the Zynq-7000 Series FPGAs, is that there is a dual core ARM core embedded next to the FPGA.
This dual core CPU can be used to run a full operating system on the development board. Part of the tutorials is 
to create a Petalinux image to run on the development board. This is done through creating a custom linux kernel with
the custom FPGA components added to be accessed from the linux operating system drivers. Example block designs
for the Ethernet FMC card were intergrated with a custom linux image to test the ethernet capabilities of the FMC 
card.

\subsection{Embedded}

\par Connecting custom kernels for peripherals made in the FPGA takes a long amount of time to compile and test. 
Another process of testing kernel FPGA extensions was to write Bare-Metal applications using the Software 
Development Kit. These are C code programs compiled to run when the FPGA programmed over the JTAG header, or an 
image can be created to run the program on boot. This project has a Bare-Metal demo which runs on the Zedboard 
with an Ethernet FMC card connected, measuring the latency found in a loopback test. 

\section{Block Design}

\par PUT IMAGE HERE

\par To construct a easy to view representation of a large system design (IMAGE REFERENCE HERE), Vivado has a block level designer for
FPGA designs called the IP Integrator. It shows interconnects between logical blocks of the design and allows for 
configuration of blocks to fit specific purposes. The IP Integrator separates the Processing System (PS) and the 
Programmable Logic (PL) of the FPGA. This is needed as the PS is represented as a reconfigurable block and cannot be redesigned, while
the PL can only consist of FPGA specific blocks which can connect to the PS. 

\subsection{Zynq Processing System (PS)}
\subsection{Zynq Programmable Logic (PS)}
\subsection{Advanced Microcontroller Bus Architecture (AMBA)}
\subsection{Advanced eXtensible Interface (AXI)}
\subsection{Clock Generation}
\subsubsection{Phased Locked Loop}
\subsection{Packet Generator}
\subsubsection{RGMII Signal Processing}
\subsection{Packet Detector}
\section{Software Development Kit}
\subsection{Reading AXI Registers}
\subsection{SD Card}
\subsection{Retrieving Test Results}
