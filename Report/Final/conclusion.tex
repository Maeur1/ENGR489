\chapter{Future Work and Conclusion}\label{C:conc}

As the device is a minimum viable product, there are many improvements that can be made to make the device more 
reliable and flexible. 

\section{ID Packets}

Currently the device has a standard packet format which is hard coded into the FPGA fabric, to produce network 
packets. The idea of producing packets with increasing IDs can ensure that tests can be run in parallel and would 
not interfere with each other. This is the hypothesized reason for the short latencies produced during testing. This 
would help ensure that each test is independent and could also remove the problem of very short outliers in the 
results.

\section{Connect to the Internet for added Flexibility}

The device made has a spare ethernet port to connect to an internet connected network. This would allow the device 
results to be published to cloud based information aggregators such as AWS Kinesis.

\section{Remote Connectivity}

Connecting the device to a network, would allow an operator to connect to the device remotely and run tests without 
being physically present with the device. This would be advantageous in a datacentre setting, as locations of 
servers are commonly a harsh environment for humans to be in. To allow the remote connection to the device, a Linux 
image would need to be created with a Linux program to run the tests. 

\section{Custom Hardware}

The Zedboard and Ethernet FMC card both contain extra hardware which is not utilized in this project. A custom PCB 
with the FPGA would be able to significantly reduce the cost of entry to this platform by mass manufacturing a PCB 
with only the required components on board.

\section{Conclusion}

The goal of this project is to create a device which can allow for flexible, high precision network
latency measurements. Current solutions are either flexible or not precise enough to allow for low
latency measurements.

It was found that an FPGA would be a suitable device for the task, as the complexity of allowing a
flexible platform while also providing high speed low level access to electrical signals. The device
platform for the design is a Zedboard by AvNET which features a XC7Z020-CLG484 FPGA.

The device is currently a minimum viable product. The device can provide high resolution and
reliable timings in a flexible format. This is useful for new research being done with high speed
communications as the need for faster data transfer grows and the measurement of smaller and
smaller latencies is becoming more of an issue. This is currently solved by using one of the many
hardware device platforms that are available, but these do not offer flexible way of obtaining results
and require training in using the device. Software platforms provide the flexibility that hardware
systems do not provide, but produce inconsistent and unreliable results.

As shown by this device, hardware platforms can be both flexible and provide reliable testing results.
This device fills a gap in the market where no proprietary software is needed to accompany the
device for network packet analysis.
