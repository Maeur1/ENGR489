\chapter{Introduction}\label{C:intro}

\par Fast internet connections are becoming more and more desired as high bandwidth media consumption and internet based services grow in popularity. 
Speed of an internet connection can be separated into two distinct metrics, latency and bandwidth. 
Latency is the time taken for information to travel from one place to another and the delay introduced by the device itself, while bandwidth is the amount of information that can flow reliably from end to end.
When a user is accessing a website, such as Facebook, their requests traverse multiple network devices
(physical hardware) to reach that websites’ servers. Throughout this travel, each network device
introduces some latency, as there is processing time required to send information to the right
destination. Although this delay is very small, over the course of multiple jumps it can become
significant.

\section{Problem}

\par Current software solutions can measure latency in ethernet network traffic down to a resolution of
milliseconds and sometimes microseconds \cite{pingisbad}. There are no software solutions that can measure
latency in the order of nanoseconds reliably \cite{timeinlinux}. Measuring the latency in nanoseconds is needed as
network switches have very low latency and the time between ingress and egress can range from
microseconds to nanoseconds. If it could be measured accurately, then analysis could be performed
on network traffic through network switches to research new ways to reduce latency.

\par A reduction in network latency would be a huge boon to future technologies. For example, less
latency in video streaming can enable doctors from remote locations to perform surgery with
minimal lag between actions and responses \cite{remotesurgery}. The stock market is another example of improvement with
this technology, as less latency ensures that stock market trade deals can be completed faster.

\begin{quote}
    \centering
    ``We are running through the United States with dynamite and rock saws, so that an algorithm can
    run three microseconds faster.`` \em --- Kevin Slavin, How Algorithms shape our world \cite{tedTalkAlgorithms}
\end{quote} 

Many improvements made at a nanosecond scale cam add up to microseconds or even milliseconds of latency
reduction. Improvements cannot be done without knowing how much this can improve by without the tools to measure it.

\section{Solution}

\par A software based approach has the limitation that the timing functions must be processed by the
CPU. This means the counters for timing functions could be offset from the true value. Higher
resolution latency measurements can be achieved through a hardware implementation of a packet
timer. This circumvents the limitations of a software based approach.
Latency would be best measured by considering the raw electronic signals from the ethernet port on
the device. To achieve this, microcontrollers, discrete logic gates and a Field Programmable Gate
Array could be implemented. A microcontroller could be used but then the same issues arise, that a
CPU clocking through instructions would deviate the true timing counter value. Another approach
would be to use discrete logic gates, but this will become complicated very rapidly, and propagation
delays through discrete devices would reduce the accuracy. Hence the preferred choice is to
implement the packet latency measurement unit on a Field Programmable Gate Array (FPGA). More
information about FPGA’s is in the Background section of the report. This way complex digital
designs can be simplified down to a simple set of blocks, and there is no overhead that a processor
would introduce.

\par This report will focus on the work done so far in implementing a FPGA based packet latency timing
device. This project is limited to ethernet based networks as they are most widely used \cite{etherneteverywhere}.
