\chapter{Background}\label{C:back}

\section{Network packet}

\par Communication between computers is established through the means of physical connections and protcols overlaying on top of this. 
A widespread communication protocol is to break down each point of communication into discrete pieces called network packets.
These network packets contain many layers (Figure: \ref{fig:OSIModel}) of communication information.
Communication layers can be broken down into a few independent layers which hold specific purposes. 
Each layer contains information which can aid with transporting the information to the correct computer securely.
As each layer is independent, each higher layer does not affect the forthcoming layers.
High layers of the model are for purposes such as reliability of data and the data itself in the packet.
For the purposes of this project, an explanation of only the first three layers of the Open Systems Interconnection (OSI) model is concerned.

\begin{figure}[H]
    \begin{center}
        \includegraphics[width=5cm,height=5cm,keepaspectratio]{Images/OSIModel.png}
        \caption{OSI Model}
        \label{fig:OSIModel}
    \end{center}
\end{figure}

\subsection{Layer 1: Physical}

\par The physical layer is specifying the electrical and physical medium that is used to transport individual bits from one computer to another.
Two or more computers can be connected through many different physical mediums such as copper wiring and light fibers. 
Each medium has specific pros and cons, from implementation costs to throughtput capabilities.
The physical layer is responsible for transmitting and receiveing unstructured data.
Communication can be either Simplex, Half Duplex or Full Duplex.
The network topology is defined by how the physical layer is configured.
An example of a phyiscal medium is a copper medium is used to link two devices together.
This project will be investigating the latency present in a Gigabit Ethernet system, using copper based medium.

\subsection{Layer 2: Data link}

Data link layer is responsible for node-to-node data communication. 
It links from one device to another, determing the connection status of two physically connected devices.
A feature of this layer, is the ability to detect and potentially correct errors created at the physical layer.
\cite{IEEE802}

\subsection{Layer 3: Network}

\section{Existing Solutions}

\subsection{DPDK}

\subsection{PF\textunderscore RING}

\section{Measurement Techniques}

\subsection{Cut-Through}

\subsection{Store-and-Forward}

\section{Field Programmable Gate Array}
