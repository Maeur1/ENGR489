\chapter{Work Done So Far}\label{C:workDoneSoFar}

\section{FPGA Design Creation}

\par To work with the specified FPGA (Zynq Z7020 \cite{fpga}) a set of proprietary software solutions must be used to develop designs.
These are a suite of software from Xilinx, which are under Vivado Design Suite, the Software Development Kit and also the Petalinux software.
The process of creating a custom design on the FPGA requires code to be written in Hardware Descriptor Language (HDL).
I have created a simple design where it would take a on board clock and generate a signal which makes an LED flash every second using logic gates.
This helped me learn the design suite, and also understand the workflow to get a design onto the FPGA.

\section{Xilinx University Program}

\par Initially, I followed the Xilinx University Program files which were supplied to me by Robin Dykstra.
These tutorials taught me how designs can be synthesised and implemented on the FPGA, and the constraints that I had when making my own design.
I began by learning about how to create a completely independent FPGA fabric design, which does not incorporate any features present in the ARM core processors.
I followed on to creating designs which used the ARM cores to produce a linux subsystem.
This learning would be useful when producing a user friendly interface to initiate a test, and retrieve the results.
Following the linux subsystem, I ended with a complete design where the linux subsystem would interact with FPGA fabric through an interconnect called Advanced Microcontroller Bus Architecture (AMBA).

\section{Petalinux}

\par Petalinux is a section of the software suite provided by Xilinx.
It provides the tools to create a custom linux kernel to incorporate features in the FPGA fabric.
This is vital to the project, as the timing of the ethernet packets would be done in the FPGA fabric, and the data would be accessed in the Linux system.

\par I worked on producing a custom petalinux image where the linux would detect the ethernet interface added by the Zedboard, and be able to communicate through it.
This took a considerable amount of time, as the information available in the internet was conflicting, or was outdated.
I eventually was able to create a petalinux image which could bring up the ethernet interface, and allow traffic to flow through the default kernel driver.

\section{PicoZed Development Board}

\par Initially, A PicoZed development board with a FMC carrier V1 was planned as the base of the project.
Half way through the implementation of the design, it was realised that it was not suitable for the project.
This was because the Low Pin Count (LPC) FPGA Mezzanine Card (FMC) connector on the FMC carrier card was incorrectly routed to the wrong pins on the FPGA.
The FPGA has specific pins for allowing clock signals to pass through, called Clock Capable (CC) Pins.
These pins have buffers on the FPGA designed to handle fast clock signals.
The FMC Carrier card V1 did not allow for the correct Clock Capable pins to be routed to the on board ethernet phy.
When I began testing the design, the design could not be synthesized because of this restriction, hence I changed development board to the Zedboard.

\section{Ethernet FMC card}

\par Once the Zedboard was set up, there was minimal change to the code.
But another problem was encountered, which was that the Ethernet FMC card would prevent the Zedboard from booting.
This investigation is ongoing, and a new Ethernet FMC card is currently being sent to me, but it is unknown why the current card would cause these problems.
I have a Petalinux image ready to accept the extra Ethernet phys on the Ethernet FMC card, and once the new one has arrived, the work can continue.

\section{C Implementation}

\par As there were issues with the FPGA implementation, I wrote a C implementation which gets timestamps from the kernel directly.
This was a temporary measure so that some results can be retrieved if the FPGA implementation gets too far behind.
