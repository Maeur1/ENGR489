\chapter{Work Done So Far}\label{C:workDoneSoFar}

\par The FPGA chosen to implement a packet timing unit is a XC7Z020 Zynq-7000 FPGA by Xilinx. This
FPGA was chosen as it is a modern part that has a large amount of supporting documentation online
and expertise for the platform is locally available at the University, should I need some assistance
with the project.

\par To develop for this FPGA, a development board is required as there are many additional components
that are needed for the development of this project. This includes flash memory, Random Access Memory
(RAM), an Universal Asynchronous Receiver/Transmitter (UART) and an Ethernet Physical Transceiver
(Ethernet Phy). Initially the project was to be implemented on a PicoZed Z7020 System-On-a-Module
(SOM) combined with a PicoZed FMC Carrier V1, for its dual on board ethernet port capabilities (one
can serve as input, other as output), but was later changed to a Zedboard for reasons discussed
further.

\par Initial work began with learning the software workflow designed by Xilinx. This consisted of
understanding a hardware descriptor language (HDL) supported by Xilinx and the complementing
toolchain. I had experience with both VHISC Hardware Descriptor Language (VHDL) and Verilog, two
HDL's supported by Xilinx, so little time was needed learning the language.

\section{Xilinx University Program (XUP)}

Xilinx provides FPGA toolchain tutorials from which I learned how to construct a basic FPGA design in
a software called Vivado. The tutorials in the program were targeted to a Zedboard (Different FPGA
development board), so modifications had to be done to ensure the design was compatible with the
PicoZed Z7020 SOM on the PicoZed FMC Carrier V1.

\section{Implementation Stage}

\par I began implementing the second Ethernet Phy on the PicoZed FMC Carrier V1, where the first
problem was encountered. The second Ethernet Phy on the development board requires a clock
signal for transmission of data from the FPGA, and the Ethernet Phy generates a clock signal for data
reception to the FPGA. Both clock signals were incorrectly wired to non-clock capable pins on the
FPGA itself because the PicoZed FMC Carrier V1 was physically manufactured incorrectly.

\par A solution to this problem was to use an Ethernet FPGA Mezzanine Card (FMC) which adds 4 more
Ethernet Phys onto a development board but the same problem occurred as before. The FMC
connector on the development board required specific clock capable pins for the Ethernet FMC to be
routed to specific FPGA pins. This was not correctly done and two of the 4 Ethernet Phys would be
non-functional.

\par This myriad of problems suggested that a change in development board would be appropriate,
hence the Zedboard was chosen. The Zedboard has the same XC7Z020 FPGA with a single on-board
Ethernet Phy but the clock capable pins were correctly routed to the FMC connector for the Ethernet
FMC to expand the total number of Ethernet Phys to five.

\par Progress made throughout this problem searching phase was to ensure that the platform is ready for implementation, and ensure compatibility with components. This was more of a planning stage to ensure that the execution of the project can be continued without interruptions.

\section{Current Position}

\par Currently the Ethernet FMC block design (Appendix: \ref{appendix:blockdesign}) has been compiled, and work that needs to be done is to integrate in custom logic for the timer. The value from the timer can be retrieved by the ARM based Linux subsystem and saved to a file for future analysis. 

\par It was recently found that the Ethernet FMC card would cause a failure to boot on the Zedboard. The
Zedboard would function with the Ethernet FMC card installed, but when installed, the boot process
on the Zedboard was unknowingly halted. After contacting the creator of the Ethernet FMC, it was
concluded that the FMC card was faulty and a replacement is currently shipped and on its way to me.

\par I have written a C program which does a microsecond proof-of-concept program which can be used
to gather data if the FPGA design does not get implemented in time. It measures the time taken for a
packet generated in C to traverse out a network interface card (NIC) and then listen on another NIC
to receive the packet and measure the time difference between sending and receiving. This program
achieves the same level of accuracy as any software based implementation (order of microseconds).
